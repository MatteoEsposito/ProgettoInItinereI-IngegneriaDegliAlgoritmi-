\chapter{Analisi delle Prestazioni: Caso Peggiore}%: Ricordati di inserire anceh il caso uno dei due vuoto}
\section{Analisi Teorica}
L'analisi delle prestazioni, nel caso peggiore, consente di fatto di stabilire\'e quanto sia pi\'u o meno performante un dato algoritmo. L'analisi di questo algoritmo in particolare consta di diversi snipset di codice che contengono prevalentemente operazioni il cui tempo di esecuzione sia pari all'$O(log(n))$.
\newline
\newline

\begin{minipage}{0.49\linewidth}
	\begin{Verbatim}[frame=topline,numbers=left,label=Codice,framesep=3mm]
	H_a = A.getTreeHeight()
	H_b = B.getTreeHeight()
	if H_a <= H_b :
		# Codice Caso H_a <= H_b
	else:
		# Codice Caso H_a > H_b
	\end{Verbatim}
\end{minipage}\hfill
\begin{minipage}{0.49\linewidth}
	\begin{Verbatim}
	La funzione Concatenate chiama un'altra 
	funzione che ritorna l'altezza della radice
	Ottenute le due altezze fa un brevissimo
	confronto per stabilire in quale caso 
	ricade l'attuale esecuzione quindi procede
	con il blocco di codice adatto.
	\end{Verbatim}

\end{minipage}
\newline \newline % Confronto Alberi

\begin{minipage}{0.49\linewidth}
	\begin{Verbatim}[frame=topline,numbers=left,label=Codice,framesep=3mm]
	def getTreeHeight(self):
	    return self.height(self.tree.root)
	\end{Verbatim}
\end{minipage}\hfill
\begin{minipage}{0.8\linewidth}
	\begin{Verbatim}
	La funzione getTreeHeight, 
	sovrabbodnate rispetto alla funzione
	height, restituisce in Tempo Costante
	O(1) l'altezza dell'albero ritornando
	essenzialmente l'altezza della radice
	\end{Verbatim}
\end{minipage} % Funzione getTreeHeight
\newline 

\paragraph{Attenzione:} per coerenza e continuità con quanto esposto prima nella descrizione dell'algoritmo continueremo con l'ipotesi che l'albero B sia più alto dell'albero A \newline\newline


\begin{minipage}{0.49\linewidth}
	\begin{Verbatim}[frame=topline,numbers=left,label=Codice,framesep=3mm]
	R_a = A.getMaxNode()
	R_av = A.value(R_a)
	R_ak = A.key(R_a)
	A.delete(A.key(R_a))
	\end{Verbatim}
\end{minipage}\hfill
\begin{minipage}{0.49\linewidth}
	\begin{Verbatim}
	Ora viene estratto il massimo da A
	Ne  viene salvato il valore
	ed  il valore key
	e viene elimianto da A
	\end{Verbatim}
\end{minipage}
 \newline\newline % Ricerca Massimo


 \begin{minipage}{0.5\linewidth}
   	\begin{Verbatim}[frame=topline,numbers=left,label=Codice,framesep=5mm]
   def getMaxNode(self):
        curr = self.tree.root
        while curr.rightSon != None:
            curr = curr.rightSon
        return curr
        
    \end{Verbatim}
    \end{minipage}\hfill
\begin{minipage}{0.405\linewidth}
\ttfamily
La funzione getMaxNode percorre la parte destra dell'albero fino a restituire il nodo più a destra Nella Peggiore delle Ipotesi, questo metodo consta di un'esecuzione temporale pari all'\emph{O(log(n))}, data che la peggiroe delle  ipotesi è che debba scendere per l'intero albero il quale, per via delle prorpietà degli AVL  ha un'altezza pari a log(n)

\end{minipage}
\newline	
\newline % Funzione getMaxNode

 \begin{minipage}{0.5\linewidth}
	\begin{Verbatim}[frame=topline,numbers=left,label=Codice,framesep=5mm]
   R_b = B.searchLeftForNodeOfHeight(H_a)
    R_bt = B.tree.cut(R_b)
        \end{Verbatim}
\end{minipage}\hfill
\begin{minipage}{0.4\linewidth}
\begin{Verbatim}
Viene quindi ricercato nella parte sinistra 
dell'albero B un nodo che abbia altezza 
eguali o di una singola unità più alto, 
nell'ipotesi peggiore, quindi si stacca
dall'albero con il suo intero 
relativo sotto albero

\end{Verbatim}

\end{minipage} % Ricerca del Nodo R

\begin{minipage}{0.5\linewidth}
   	\begin{Verbatim}[frame=topline,numbers=left,label=Codice,framesep=5mm]
    def searchLeftForNodeOfHeight(self, height):
        curr = self.tree.root
        while self.height(curr) != height 
        and self.height(curr) != height+1: 
            if curr.leftSon  == None:
                break
            else:
                curr = curr.leftSon
        if self.height(curr.leftSon) == height: 
            return curr.leftSon                 
        else:                                   
            return curr
        \end{Verbatim}
\end{minipage}\hfill
\begin{minipage}{0.395\linewidth}
\ttfamily
La funzione searchLeftForNodeOfHeight ricerca a sinistra di un dato albero un nodo che possegga altezza eguali o di una singola unità in più rispetto allìaltezza chiesta. Nella peggiore delle ipotesi l'algoritmo scenderà fino ad una foglia, in tal caso, il tempo di esecuzione di tale istanza sarà pari all'altezza dell'albero stesso che, grazie alle proprietà degli Alberi AVL, sar\'a pari al log(n)
\end{minipage}
\newline
\newline % Funzione searchLeftForNodeOfHeight

 \begin{minipage}{0.45\linewidth}
	\begin{Verbatim}[frame=topline,numbers=left,label=Codice,framesep=5mm]
 tempA = createAVLByArray([R_av])
 tempA.tree.insertAsLeftSubTree(tempA.tree.root,A.tree)
 tempA.tree.insertAsRightSubTree(tempA.tree.root, R_bt)
 tempA.updateHeight(tempA.tree.root)
     \end{Verbatim}
    \end{minipage}\hfill
\begin{minipage}{0.47\linewidth}
\begin{Verbatim}
	Ora l'agoritmo crea un nuovo albero 
	temporaneo, la funzione 
	createAVLByArray, nella peggiore 
	delle ipotesi impiega un tempo lineare 
	all'input, ma in questo particolare caso 
	la peggiore delle ipotesi è irrealizzabile
	infatti creando un albero di un singolo
	elemento, impiegherà tempo costante. 
	O(1)) l'inserimento come figlio sinistro
	e destro invece  constano anch'essi 
	di operazioni di tempo costante.
	Quindi ne aggiorna l'altezza, tale 
	operazione può essere fatta in tempi
	costanti ma nella peggiore delle ipotesi
	può richidere tempo logaritmico

\end{Verbatim}
\end{minipage}
\newline
\newline % Creazione dell'albero temporaneo

 \begin{minipage}{0.5\linewidth}
 \begin{Verbatim}[frame=topline,numbers=left,label=Codice,framesep=5mm]
 B.tree.insertAsLeftSubTree(FtR_b, tempA.tree)
 B.balInsert(FtR_b.leftSon)
  \end{Verbatim}
\end{minipage}\hfill
\begin{minipage}{0.39\linewidth}
\begin{Verbatim}
Viene quindi inserito come 
figlio sinistro al padre del nodo R
l'intero albero temporaneo 
precedentemente creato
quindi si chiede di 
ribilanciare verso l'alto
tale operazione nella peggiore delle
ipotesi impiegherà un tempo di esecuzione
logaritmico
\end{Verbatim}
\end{minipage}
\newline
\newline % Innesto dell'albero Temporaneo

\paragraph{Riepilogo:}
Il codice così analizzato ha evidenziato il fatto che qualsiasi altra chiamata a funzioni nuove da me create per estender ancora di più la classe AVLDict, nello scenario della funzione principale \emph{concatenate} risultano costare tutte tempo logaritmico. Quindi l'analisi teorica ci conferma che l'algoritmo asintoticamente, quindi per grandi quantit\'a di dati, è un $O(log(n))$

% \begin{minipage}{0.5\linewidth}
\begin{Verbatim}[frame=topline,numbers=left,label=Codice,framesep=5mm]
H_a = A.getTreeHeight()
H_b = B.getTreeHeight()

R_a = A.getMaxNode()
R_av = A.value(R_a)

A.delete(A.key(R_a))

R_b = B.searchLeftForNodeOfHeight(H_a)

if R_b.father == None:
    isRoot = True
    FtR_b = B.tree.root
else:
    isRoot = False
    FtR_b = R_b.father

R_bt = B.tree.cut(R_b)

tempA = createAVLByArray([R_av])
tempA.tree.insertAsLeftSubTree(tempA.tree.root,A.tree)
tempA.tree.insertAsRightSubTree(tempA.tree.root, R_bt)

tempA.updateHeight(tempA.tree.root)
B.tree.insertAsLeftSubTree(FtR_b, tempA.tree)

B.balInsert(FtR_b.leftSon)

return B
    
  \end{Verbatim}
\end{minipage}\hfill
\begin{minipage}{0.39\linewidth}
\begin{Verbatim}
0(1)
0(1)

0(1)
0(1)


0(1)

0(log(n))











0(log(n))
0(1)


0(log(n))
\end{Verbatim}
\end{minipage}
\newline
\newline % Riepilogo
